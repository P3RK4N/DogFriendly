\chapter{Zaključak i budući rad}

        Tema našeg projektnog zadatka bila je izrada web aplikacije pod imenom Dog Friendly. Bit same aplikacije je bilo napraviti interaktivnu kartu kako bi ljubiteljima i vlasnicima pasa omogućili pregled prikladnih i neprikladnih lokacija na interaktivnoj karti i time olakšali kretanje i druženje. Izrada cijelog projekta je trajala 14 tjedana i bila je podijeljena u dva ciklusa.

        Napredak je u prvom ciklusu bio nešto sporiji, a glavni razlog tome bilo je prvotno okupljanje tima, međusobno upoznavanje, upoznavanje sa samom temom projekta i upoznavanje s novim alatima koje smo trebali koristiti u izradi projekta. Nama je ta početna faza upoznavanja trajala relativno kratko zato što su svi članovi tima odmah shvatili važnost ovog projekta i znali su koji cilj trebamo postići. U timu smo se organizirali u dvije skupine, frontend i backend, što nam je kasnije uvelike pomoglo zato što su svi članovi znali svoju ulogu i znali su koji će ih posao kasnije čekati. Ta prva faza se fokusirala većinom na izradu projektne dokumentacije, a manje na izradu same aplikacije. Najveći tehnički izazovi koji su se javljali tada su bili korištenje Git-a (distribuirani sustav za upravljanje izvornim kodom) i korištenje LaTeX-a (programski jezik za pisanje dokumentacije), no te smo izazove savladali s vremenom kada smo se svi zajedno malo bolje upoznali s tim alatima. Najbitnija uloga uspješne izrade prvog dijela projekta bila je uloga voditeljice tima koja nas je konstantno obavještavala o našem napretku na projektu i koja nam je svima zadavala zadatke koji su nam bili u planu izrade. Dobro postavljeni temelji prve faze projekta poput kvalitetne izrade obrazaca uporabe i dobre organizacije tima su nam uvelike pomogli u izradi iduće faze projekta.
        
        U drugom ciklusu je veći fokus bio na implementaciji same aplikacije. U ovoj fazi su svi članovi imali potpunu samostalnost nad svojim zadacima, svi članovi frontenda i backenda su bili u komunikaciji u dijelovima gdje je to bilo potrebno zato što je tako bilo najlakše ispuniti sve zahtjeve koje je aplikacija morala zadovoljiti. Tehnički izazovi koji su nam ovdje predstavljali najveće probleme su bili izrada interaktivne karte na frontendu i izrada programskih ispita na backendu pomoću kojih bi ispitali ponašanje nekog dijela sustava i pojedinih komponenti koje implementiraju neke temeljne funkcionalnosti aplikacije. Ti problemi su se javljali zbog neiskustva članova tima u izradi tih specifičnih zadataka, no te smo probleme uspješno riješili uz pomoć asistentice koja je cijelo vrijeme nadzirala naš napredak na projektu. Uspjeh druge faze projekta bio je zbog dobre organizacije posla među članovima.
        
        Znanja koja smo stekli na ovom projektu su mnogobrojna. Neka od praktičnih znanja su pisanje programske dokumentacije, korištenje UML modeliranja u projektu, oblikovanje arhitekture programske potpore prema objektno orijentiranoj paradigmi, analiza korisničkih zahtjeva i korištenje različitih razvojnih okruženja za izradu aplikacije. Također smo stekli mnogo iskustvenog znanja kao što su timski rad, organizacija i vođenje samog projekta, međusobna komunikacija svih članova tima, razmjenjivanje različitih mišljenja u pronalasku zajedničkog rješenja i mnoga druga. Jedino znanje koje bi bilo korisno znati prije samog početka izrade projekta bi bilo korištenje Git-a.
        
        Ostvarili smo sve zahtijevane funkcionalnosti koje su se od nas tražile tako da možemo smatrati ovaj projekt uspješnim. Jedna od mogućih funkcionalnosti koje nismo implementirali, ali smatramo da bi mogla poboljšati izgled aplikacije bi bila mogućnost dodavanja slike obrta te bi se te slike onda prikazivale pod sekcijom preporučenih obrta. Osim svih funkcionalnosti koje smo ostvarili također smo naučili da je međusobna komunikacija u timu izuzetno važna što je i bio sami cilj projekta. Sve u svemu zadovoljni smo izradom našeg projekta i veselimo se upotrijebiti stečeno znanje i na neke druge projekte.


		\eject 